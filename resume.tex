% !TEX program = xelatex

\documentclass{resume}
%\usepackage{zh_CN-Adobefonts_external} % Simplified Chinese Support using external fonts (./fonts/zh_CN-Adobe/)
%\usepackage{zh_CN-Adobefonts_internal} % Simplified Chinese Support using system fonts
\usepackage{hyperref}
\usepackage{fancyhdr}


% \pagestyle{fancy}
% \fancyhead[C]{\rule{.5\textwidth}{4\baselineskip}}% Add something BIG in the header
% \setlength{\headheight}{120pt}
% \chead{
%   \name{Ende Jin}
%   \basicInfo{
%   \email{ende.jin@mail.utoronto.ca} \textperiodcentered\ 
%   \phone{(+1) 647-6854680} 
% }
% \basicInfo{
%   \linkedin[Ende Jin]{https://www.linkedin.com/in/ende-jin-dkxxl/} \textperiodcentered\ {https://www.linkedin.com/in/ende-jin-dkxxl/}
% }
% \basicInfo{
%   \github[EDJ]{https://github.com/DKXXXL} \textperiodcentered {https://github.com/DKXXXL}
% }
% }
% \rhead{\thepage}

\begin{document}

\name{Ende Jin}

\basicInfo{
  \email{ende.jin@mail.utoronto.ca} \textperiodcentered\ 
  \phone{(+1) 647-6854680} 
}

\basicInfo{
  \linkedin[Ende Jin]{https://www.linkedin.com/in/ende-jin-dkxxl/} \textperiodcentered\ \url{https://www.linkedin.com/in/ende-jin-dkxxl/}
}
\basicInfo{
  \github[EDJ]{https://github.com/DKXXXL} \textperiodcentered \url{https://github.com/DKXXXL}
}

\section{\faGraduationCap\ Education}
\datedsubsection{\textbf{University of Toronto}}{2016 -- present}
\textit{Honours Bachelor of Science \underline{(In Progress)}} 
\\ \textperiodcentered\  Computer Science Specialist 
\\ \textperiodcentered\  Mathematics Major

\section{\faUsers\ Experience}
\datedsubsection{\textbf{Fields Undergraduate Summer Research Program} Toronto, Canada}{2018, Summer}
\role{Summer Research Internship} {}
Project : Discrete-Event Systems Model of a System's Ability to Protect Secrets.

Creating an algorithm that comes up with minimal communication protocol 
for several agents who cooperate to discover the secret information of a machine 
which has partial information hidden. \\
This research can help a system to evaluate the current strategy 
that hiding information from several collaborating hackers.




% Reference Test
%\datedsubsection{\textbf{Paper Title\cite{zaharia2012resilient}}}{May. 2015}
%An xxx optimized for xxx\cite{verma2015large}
%\begin{itemize}
%  \item main contribution
%\end{itemize}

\section{\faCogs\ Skills}
\subsection{\textbf{Experience with Multiple Programming Languages}}
\begin{itemize}
 \item Haskell, Coq, C/C++, Scheme, Python, Java, Javascript
\end{itemize}
\subsection{\textbf{Intermediate Knowledge of Computer Science and Mathematics}}
\begin{itemize}
\item Have the basic knowledge of the process of compilation and interpretation of programming languages
\item Understand several basic compiling techniques in functional programming languages
\item Introduction of Calculus, Multivariable Calculus, Linear Algebra, Ordinary Differential Equation and Number Theory
\item Understand the generated assembly code within the help of developer manual
\end{itemize}
\subsection{\textbf{Acquisition of Two languages}}
\begin{itemize}
\item Chinese (Mandarin) – Native Speaker
\item English – Written/Spoken
\end{itemize}


\section{\faCogs\ Projects}
\begin{itemize}
\item (subset of dynamic typing) Scheme to C transpiler. References: \textit{Structure and Interpretation of Computer Program}, \textit{‘Write Yourself a Scheme in 48 Hours’} and \textit{Essentials of Programming languages}. \\
\textit{\underline{in Haskell}} \hfill \url{https://github.com/DKXXXL/emesch}
\\
\item An interpreter and compiler of variant Scheme with Static type system (subtyping, sum type, record type). An extension from the above project. References: \textit{Software Foundations} and \textit{'Implementing a JIT Compiled Language with Haskell and LLVM'} \\
\textit{\underline{in Haskell, a compiler target to LLVM}} \hfill \url{https://github.com/DKXXXL/Schmidty} \\
\textit{\underline{in Coq, a semi-verified(soundness proved) interpreter}} \hfill \url{https://github.com/DKXXXL/VSchmidty} \\
\\
\item A theorem prover based on Calculus of Construction. Designed to be extensible with tactic system. A sketchy imitation of Coq. References: \textit{Type Theory and Formal Proof}. Thanks to flow.js and a series of packages.\\
\textit{\underline{in Javascript, with flow.js}} \hfill \url{https://github.com/DKXXXL/ITP.js}
\\
\item (Partial) answers to programming exercises in \textit{Software Foundations} and \textit{Coq’Art Interactive Theorem Proving and Program Development}. \\
\textit{\underline{in Coq}} \hfill \url{https://github.com/DKXXXL/Coq-Art-Exercises} \\
..  \hfill \url{https://github.com/DKXXXL/SoftwareFoundations-BeforeCh15} \\
..  \hfill \url{https://github.com/DKXXXL/SoftwareFoundations-AfterCh15} \\
\\
\item Implementing gobang on Altera DE2-115 board. Final Project for CSCB58S17. \\
\textit{\underline{in Verilog}} 
\hfill \url{https://github.com/DKXXXL/FiveSons\_Mirror}

\end{itemize}

% \section{\faHeartO\ Honors and Awards}
% \datedline{\textit{\nth{1} Prize}, Award on xxx }{Jun. 2013}
% \datedline{Other awards}{2015}

\section{\faInfo\ Miscellaneous}
\begin{itemize}[parsep=0.5ex]
  \item Linkedin: \url{https://www.linkedin.com/in/ende-jin-dkxxl/}
  \item GitHub: \url{https://github.com/DKXXXL}
\end{itemize}

\subsection{\textbf{Interested in …}}
\begin{itemize}
\item Formal Verification, Program Analysis and Functional Programming
\item Type System and Programming Language. Type Theory, Structural Proof Theory
\item Interpreter and Compiler
\item Mathematical logic (especially Proof Theory) and Category Theory
\end{itemize}

%% Reference
%\newpage
%\bibliographystyle{IEEETran}
%\bibliography{mycite}
\end{document}
