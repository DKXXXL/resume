\documentclass[fontsize=11pt]{article}
    \usepackage[english]{babel}
\usepackage[utf8]{inputenc}
\usepackage[T1]{fontenc}
\usepackage{lmodern}
% \usepackage[protrusion=true,expansion=true]{microtype}
\usepackage[svgnames]{xcolor}  % Colours by their 'svgnames'
\usepackage[margin=0.75in]{geometry}
  \textheight=700pt
\usepackage{url}
\usepackage{lmodern} % Allow arbitrary font sizes
\usepackage{textcomp}
\usepackage{hyperref}



%% Define a new 'modern' style for the url package that will use a smaller font.
\makeatletter
\def\url@modernstyle{
  \@ifundefined{selectfont}{\def\UrlFont{\sf}}{\def\UrlFont{}}}
\makeatother
\urlstyle{modern} %% And use the newly defined style.

\frenchspacing              % Better looking spacings after periods
\pagestyle{empty}           % No pagenumbers/headers/footers

\renewcommand{\familydefault}{\sfdefault}

%%% Custom sectioning (sectsty package)
%%% ------------------------------------------------------------
\usepackage{sectsty}

\sectionfont{                 % Change font of \section command
  \usefont{OT1}{phv}{b}{n}%   % bch-b-n: CharterBT-Bold font
  \sectionrule{0pt}{0pt}{-5pt}{3pt}}

%%% Macros
%%% ------------------------------------------------------------
\newlength{\spacebox}
\settowidth{\spacebox}{8888888888}      % Box to align text
\newcommand{\sepspace}{\vspace*{1em}}   % Vertical space macro

\newcommand{\MyName}[1]{ % Name
    \Huge \usefont{OT1}{phv}{b}{n} \hfill #1
    \par \normalsize \normalfont}

\newcommand{\MySlogan}[1]{ % Slogan (optional)
    \large \usefont{OT1}{phv}{m}{n}\hfill \textit{#1}
    \par \normalsize \normalfont}

\newcommand{\NewPart}[1]{\section*{\uppercase{#1}}}

\newcommand{\PersonalEntry}[2]{
    \noindent\hangindent=2em\hangafter=0 % Indentation
    \parbox{\spacebox}{                  % Box to align text
    \textit{#1}}                      % Entry name (birth, address, etc.)
    \hspace{1.5em} #2 \par}              % Entry value

% \newcommand{\SkillsEntry}[2]{                % Same as \PersonalEntry
%     \noindent\hangindent=2em\hangafter=0 % Indentation
%     \parbox{\spacebox}{                  % Box to align text
%     \textit{#1}}                    % Entry name (birth, address, etc.)
%     \hspace{1.5em} #2 \par}              % Entry value
\newcommand{\SkillEntry}[2]{       % Same as \EducationEntry
    \noindent \textbf{#1} \hfill      % Skill Title
       % Duration
        \noindent \textit{#2} \par        % Company
}

\newcommand{\SkillsEntry}[4]{       % Same as \EducationEntry
    \noindent \textbf{#1} \hfill      % Skill Title
       % Duration
        \noindent \textit{#2} \par        % Company
    \noindent\hangindent=2em\hangafter=0 \small #4 % Description
    \normalsize \par}

\newcommand{\AwardsEntry}[2]{                % Same as \PersonalEntry
    \noindent\hangindent=2em\hangafter=0 % Indentation
    \parbox{\spacebox}{                  % Box to align text
    \textit{#1}}                    % Entry name (birth, address, etc.)
    \hspace{1.5em} #2 \par}              % Entry value

\newcommand{\EducationEntry}[4]{
    \noindent \textbf{#1} \hfill      % Study
    \colorbox{Black}{ 
      \parbox{11em}{
      \centering \color{White}#2}} \par  % Duration
    \noindent \textit{#3} \par        % School
    \noindent\hangindent=0em\hangafter=0 \small #4 % Description
    \normalsize \par}

\newcommand{\WorkEntry}[4]{       % Same as \EducationEntry
    \noindent \textbf{#1} \hfill      % Jobname
    \colorbox{Black}{%
      \parbox{11em}{%
      \centering\color{White}#2}} \par   % Duration
        \noindent \textit{#3} \par        % Company
    \noindent\hangindent=0em\hangafter=0 \small #4 % Description
    \normalsize \par}

\newcommand{\ProjectEntry}[4]{         % Similar to \EducationEntry
    \noindent \textbf{#1}  \hfill {#2} \par
    \noindent \textit{#3} \par
    \noindent \small #4 % Description
    \normalsize \par}

\newcommand{\AwardEntry}[4]{         % Similar to \EducationEntry
    \noindent \textbf{#1} \noindent \textit{#3} \hfill {#2} \par
    \noindent \small #4 % Description
    \normalsize \par}
\begin{document}


    
\MyName{Ende Jin}
\bigskip
{\small \hfill ende.jin@mail.utoronto.ca | (+1) 6476854680 | \url{https://github.com/DKXXXL/} \par
 \hfill  \url{https://www.linkedin.com/in/ende-jin-dkxxl/} | \url{https://dkxxxl.github.io/}}
%%% Education
%%% ------------------------------------------------------------
\NewPart{Education}{}
\EducationEntry
{BS Computer Science Specialist + Mathematics Major}
{ \texttt{Sep 2016 - July 2021} }
{University of Toronto, Toronto.    \bf{GPA 3.92}}
%%% Work experience
%%% ------------------------------------------------------------

\NewPart{Experience}{}

\WorkEntry
{Research Assistant}
{\texttt{July 2018 - Sep 2018}}
{Fields Institute, Toronto}
{\begin{itemize} \itemsep -1pt
    \item Project : Discrete-Event Systems Model of a System’s Ability to Protect Secrets. 
    \\ Creating an algorithm that comes up with minimal communication protocol for several agents who cooperate to discover the secret information of a machine which has partial information hidden. This research can help a system to evaluate the current strategy that hides information from several collaborating hackers.
    \end{itemize}}
    
\WorkEntry
{Assistant Software Engineer}
{\texttt{May 2019 - May 2020}}
{Huawei Research Center, Toronto}
{\begin{itemize} \itemsep -1pt
    \item Work on LLVM midend and backend, target to advanced VLIW architecture.
    \item Research and attempt to deploy cutting-edged academic technology.  
	\end{itemize}}
%%% Skills
%%% ------------------------------------------------------------
\NewPart{Skills}{}
\SkillEntry{Experience of Various Programming Languages}{Coq, Haskell, Java, C/C++, Python}

\SkillsEntry{Passion and Knowledge about Both Computer Science and Mathematics}{}{}{
    \begin{itemize} \itemsep -1pt
        \item Basic knowledge of the process of compilation and interpretation of (functional) programming languages;
        Understand the generated assembly code with the help of developer manual
        \item (Multivariable) Calculus, Linear Algebra, Ordinary Differential Equation and Number Theory; 
        Computability theory and Mathematical logic and Formal Verification
        \item Self-learning Category Theory and Type Theory (Notes on the Blog)
    \end{itemize}
}

\SkillEntry{Acqusition of Two Languages}{
    Mandarin (Native Speaker) ; English (Written/Spoken)
}

%%% Projects
%%% ------------------------------------------------------------
\NewPart{Personal Projects}{}

\ProjectEntry{Scheme-subset to C Transpiler}{\url{https://github.com/DKXXXL/emesch}}
{Haskell, Text.Parser.Combinators}
{ \texttt{References}: \textit{Structure and Intrerpretation of Computer Program}, \textit{Write Yourself a Scheme in 48 Hours}, \textit{Essentials of Programming languages}}

\ProjectEntry{Typed-Scheme-like  Compiler and semi-Verifed Interpreter }{\url{https://github.com/DKXXXL/Schmidty}}
{Haskell, Coq, LLVM, llvm-pretty \hfill {\normalfont \url{https://github.com/DKXXXL/VSchmidty}}}
{An interpreter and A Compiler of a programming language, a variant of Scheme. With Sum, Record type and subtyping mechanism.  
The operational semantic is paritally verified by Coq. 
\texttt{References}: \textit{Implementing a JIT Compiled Language with Haskell and LLVM}, \textit{Software Foundations}, \textit{Essentials of Programming languages} }

\ProjectEntry{A theorem prover based on Calculus of Construction.}{\url{ https://github.com/DKXXXL/ITP.js}}
{flow.js, Javascript, Parsimmon, jsverify }
{Designed to be extensible with tactic system. 
A sketchy imitation of Coq. 
\texttt{Reference}: \textit{Type Theory and Formal Proof}
}


\ProjectEntry{(Partial) answers to programming exercises in \textit{Software Foundations} and \textit{Coq’Art}.}{}{Coq}{
\begin{itemize}
    \item \url{https://github.com/DKXXXL/Coq-Art-Exercises}
    \item \url{https://github.com/DKXXXL/SoftwareFoundations-BeforeCh15}
    \item \url{https://github.com/DKXXXL/SoftwareFoundations-AfterCh15} 
\end{itemize}
}

\ 
\end{document}